\scnheader{Программный вариант реализации логической машины интерпретации логических sc-моделей компьютерных систем}
\scnidtf{SCL-machine}
\scnidtf{ostis-inference}
\begin{scnrelfromset}{декомпозиция программной системы}
    \scnitem{База знаний SCL-machine}
    \scnitem{Решатель задач SCL-machine}
    \scnitem{Интерфейс SCL-machine}
\end{scnrelfromset}

\begin{scnrelfromset}{реализованные логические связки}
    \scnitem{импликация*}
    \scnitem{дизъюнкция*}
    \scnitem{конъюнкция*}
    \scnitem{отрицание*}
\end{scnrelfromset}

\begin{scnrelfromset}{не реализованные логические связки}
    \scnitem{эквиваленция*}
    \scnitem{строгая дизъюнкция*}
\end{scnrelfromset}

\scnheader{Решатель задач SCL-machine}
\begin{scnrelfromset}{обобщённая декомпозиция}
    \scnitem{Агент прямого логического вывода}
    \scnitem{Агент обратного логического вывода}
    \begin{scnindent}
        \scntext{примечание}{Не реализовано.}
    \end{scnindent}
\end{scnrelfromset}

\scnheader{Агент прямого логического вывода}
\begin{scnindent}
    \scnrelfrom{пример входной конструкции}{\scnfileimage[40em]{images/direct_inference_input.png}}
    \begin{scnrelfromvector}{аргументы агента}
        \scnitem{\_target\_template}
        \begin{scnindent}
            \scnidtf{targetTemplate}
            \scnidtf{targetStatement}
            \scntext{пояснение}{Шаблон, успешный поиск которого показывает, что цель логического вывода достигнута и применение правил можно прекратить.}
            \scnrelfrom{описание примера}{\scnfileimage[28em]{images/target_template.png}}
        \end{scnindent}
        \scnitem{\_rule\_set}
        \begin{scnindent}
            \scnidtf{ruleSet}
            \scntext{пояснение}{Ориентированное множество, первым элементом которого является множество правил, которые применяются в первую очередь,
                а каждое следующее множество правил применяется после предыдущего. Таким образом указываются приоритеты множеств правил.}
            \scnrelfrom{описание примера}{\scnfileimage[27em]{images/rules_set.png}}
            \begin{scnindent}
                \scnrelfrom{описание примера}{\scnfileimage[37em]{images/rule_example.png}}
            \end{scnindent}
        \end{scnindent}
        \scnitem{\_argument\_set}
        \begin{scnindent}
            \scnidtf{argumentSet}
            \scntext{пояснение}{Множество тех элементов, которые должны быть подставлены как значение переменных шаблона цели.}
            \begin{scnindent}
                \scntext{пояснение}{В данном примере значением переменной \_x в шаблоне цели может быть sc-узел
                argument\_1, тогда значением \_y будет argument\_2. Также значением переменной \_x может быть
                sc-узел argument\_2, а значением \_y -- argument\_1.}
            \end{scnindent}
            \scnrelfrom{описание примера}{\scnfileimage[30em]{images/argument_set.png}}
        \end{scnindent}
    \end{scnrelfromvector}
    \scnrelfrom{пример выходной конструкции}{\scnfileimage[30em]{images/direct_inference_output.png}}
    \begin{scnrelfromvector}{обобщённый алгоритм}
        \scnfileitem{Получение параметров агента, проверка их валидности. Вызов агента;}
        \scnfileitem{Проверка, достигнута ли уже цель в базе знаний;}
        \begin{scnindent}
            \scntext{примечание}{Выполняется поиск по шаблону target template с параметрами шаблона arguments set.}
        \end{scnindent}
        \scnfileitem{Построение вектора очереди правил на основе множества правил. Цикл по всем правилам и пока не достигнута цель;}
        \begin{scnindent}
            \begin{scnrelfromset}{циклические операции}
                \scnfileitem{Получение посылки логического правила;}
                \scnfileitem{Определение типа посылки (связка конъюнкци, дизъюнкции, отрицания или атомарная логическая формула);}
                \scnfileitem{Проверка истинности посылки в зависимости от её типа;}
                \begin{scnindent}
                    \scntext{замечание}{Конъюнкция, дизъюнкция, отрицание работают нестабильно.}
                \end{scnindent}
                \scnfileitem{Генерация по шаблону следствия;}
                \scnfileitem{Добавление в дерево решений узла правила.}
                \begin{scnindent}
                    \scntext{примечание}{Смотрите пример выходной конструкции.}
                \end{scnindent}
            \end{scnrelfromset}
        \end{scnindent}
        \scnfileitem{Возврат дерева применённых правил.}
    \end{scnrelfromvector}
\end{scnindent}

\begin{scnrelfromset}{недостатки текущего состояния}
    \scnfileitem{Импликация в текущем состоянии интерпретируется не как логическая связка, а как отношение \scnkeyword{выводимости*},
    то есть импликация не возвращает логическую константу, а генерирует новые знания. Вместо этого должны использоваться правила вывода,
    например, Modes ponens и другие правила, в процессе логического вывода.}
    \scnfileitem{В структуру ответа агента входит только узел solution, а не вся структура решения.}
\end{scnrelfromset}
